\section{Introduction} \label{sec:introduction}
In the everyday life we are continuously surrounded by sounds, and usually the human brain is able to recognize what kind of sound it is based on experience. Artificial neural networks are again based on studies of the human brain, and if the artificial neurons work as the biological ones, there should be possible training a neural network recognizing sounds as well. 

Lately, immense efforts have been put into this subject with the purpose of translating voice into text, leaded by technology companies like Google, Microsoft, Amazon and so on, who develop voice controlled virtual assistants. The technology is promising, the time saving using voice commands vs. keyboard commands is potentially huge and the market is enormous. Some even claim that virtual assistants will run our lives within 20 years. \cite{Feloni} If that is right is still uncertain, but what is certain is that the technology has great potential.

In this final project we will, based on \href{https://datahack.analyticsvidhya.com/contest/practice-problem-urban-sound-classification/}{the Urban Sound Challenge}, sort sounds into classes using various classification methods. We use the same idea as when the great technology companies recognize voice, but we are going to differentiate between sounds made by \textbf{air conditioners}, \textbf{car horns}, \textbf{children}, \textbf{dogs}, \textbf{drills}, \textbf{engines}, \textbf{guns}, \textbf{jackhammers}, \textbf{sirens} and \textbf{street musicians}. In order to do that, 


